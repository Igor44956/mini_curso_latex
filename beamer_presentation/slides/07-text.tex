\section{Texto}
\subsection{Formatando Textos no Beamer}
%--------------------------------------------------------------------------%

\begin{frame}
  \frametitle{Comandos comuns para texto no Beamer}

  \begin{block}{}
      \begin{columns}
        \begin{column}[l]{5cm}
          \inputminted[fontsize=\small]{tex}{codes/06-text_commands.tex}
        \end{column}

        \begin{column}[c]{0cm}
          \line(0,0){100}
        \end{column}

        \begin{column}[r]{4cm}
          {\small
          \emph{Texto enfatizado}   \\
          \textbf{Texto em negrito} \\
          \textit{Texto em itálico} \\
          \textsl{Texto inclinado} \\
          \alert{Texto de alerta}  \\
          \textrm{Texto em romano} \\
          \textsf{Texto em sans serif} \\
          \textcolor{green}{Texto em verde} \\
          \texttt{Máquina de escrever} \\
          }
        \end{column}
      \end{columns}
  \end{block}

\end{frame}

%--------------------------------------------------------------------------%

\begin{frame}[fragile]
  \frametitle{Tamanho de Fontes}

  \begin{tabular}{|l|c|}
    \hline
      Comando & Resultado\\
    \hline
      \verb| \tiny{minúsculo} | & \tiny{minúsculo} \\
    \hline
      \verb| \scriptsize{muito pequena} | & \scriptsize{muito pequena} \\
    \hline
      \verb| \footnotesize{nota de rodapé} | & \footnotesize{nota de rodapé} \\
    \hline
      \verb| \small{pequena} | & \small{pequena} \\
    \hline
      \verb| \normalsize{normal} | & \normalsize{normal} \\
    \hline
      \verb| \large{grande} | & \large{grande}  \\
    \hline
      \verb| \LARGE{Muito Maior} | & \LARGE{Muito Maior} \\
    \hline
      \verb| \huge{Bem Grande} | & \huge{Bem Grande} \\
    \hline
      \verb| \Huge{Enorme} | & \Huge{Enorme}  \\
    \hline
  \end{tabular}

\end{frame}

%--------------------------------------------------------------------------%

\begin{frame}[fragile]
  \frametitle{Tamanho de Fontes}

  \rowcolors{1}{lightblue}{white}
  \begin{table}
    \begin{tabular}{|l|c|c|}
      \hline
        Comando & Tamanho - (padrão 12pt)\\
      \hline
        \verb| \tiny{minúsculo} | & 6pt\\
      \hline
        \verb| \scriptsize{muito pequena} | & 8pt\\
      \hline
        \verb| \footnotesize{nota de rodapé}| & 10pt\\
      \hline
        \verb| \small{pequena} | & 11pt\\
      \hline
        \verb| \normalsize{normal} | & 12pt \\
      \hline
        \verb| \large{grande} | & 17pt\\
      \hline
        \verb| \LARGE{Muito Maior} | & 20pt\\
      \hline
        \verb| \huge{Bem Grande} | & 25pt\\
      \hline
        \verb| \Huge{Enorme} | &  25pt\\
      \hline
    \end{tabular}
    \caption{Tamanho das fontes}
    \label{Tamanho das fontes}
  \end{table}

\end{frame}

%--------------------------------------------------------------------------%

\begin{frame}[fragile]
  \frametitle{Texto Verbatim}

  \begin{itemize}[<+->]
     \item Em alguns momentos necessitamos digitar algum texto do jeito que escrevemos.
     \item Mas como fazer isto no \LaTeX?
     \item No \LaTeX uma das maneiras é utilizar o comando \textbf{verbatim}.
  \end{itemize}

  \vspace{1cm}
  \visible<4>{Existe duas maneiras de usar o \textbf{verbatim}:}

\end{frame}

%--------------------------------------------------------------------------%

\begin{frame}[fragile]
  \frametitle{Usando texto verbatim}

  \begin{itemize}[<+->]
     \item Para textos de uma linha usa-se: \\
        \begin{minted}[fontsize=\scriptsize, resetmargins=true]{tex}
            \verb|Qualquer texto...|
        \end{minted}
     \item Para grande quantidades de textos:
        \begin{minted}[fontsize=\scriptsize, resetmargins=true]{tex}
          \begin{verbatim}
             Texto ...
             Texto ...
             .....
          \end{verbatim}
        \end{minted}
  \end{itemize}

  \begin{block}<3->{Nota:}
    Para o \textbf{verbatim} funcionar é necessário adicionar a opção [fragile] no frame.
    Exemplo: \mint{tex}| \begin{frame}[fragile] ...|
  \end{block}

\end{frame}

%--------------------------------------------------------------------------%

\begin{frame}[fragile]
  \frametitle{SemiVerbatim}

  \begin{block}<1->{Exemplo de código:}
     \begin{minted}[fontsize=\scriptsize, resetmargins=true]{tex}
        \begin{semiverbatim}
           Para o texto em vermelho use o comando
           \textcolor{red}{ \\textcolor\{red\}\{Vermelho\} }
        \end{semiverbatim}
     \end{minted}
  \end{block}

  \begin{block}<2->{Resultado:}
    \begin{semiverbatim}
      Para o texto em vermelho use o comando
      \textcolor{red}{ \\textcolor\{red\}\{Vermelho\}}
    \end{semiverbatim}
  \end{block}
\end{frame}

%--------------------------------------------------------------------------%

\begin{frame}[fragile]
  \frametitle{Temas de Fonte}
  \begin{itemize}
     \item<1-> Para mudar o tipo de fonte da apresentação usamos o comando:
        \mint{tex}| \usefonttheme{serif}|
  \end{itemize}

  \begin{block}<2->{Você pode escolher as seguintes opções de temas:}
    \begin{center}
       \begin{tabular}{ll}
        serif & structureitalicserif \\
        structurebold & structuresmallcapsserif
      \end{tabular}
     \end{center}
  \end{block}

  \begin{block}<3->{Nota:}
    Para mais informações sobre fontes \textbf{leia o Beamer User Guide}
  \end{block}

\end{frame}

%--------------------------------------------------------------------------%
\begin{frame}[fragile]
  \frametitle{Tamanho de Fonte}
  \begin{itemize}
     \item Para escolher o tamanho da fonte é necessário adicionar parâmetros no inicio do documento Beamer.
        \mint{tex}|\documentclass{beamer}|
     \item Exemplo:
        \mint{tex}|\documentclass[10pt]{beamer}|
     \item As opções de tamanho são:
          \begin{itemize}
             \item 10pt
             \item 11pt (Tamanho padrão)
             \item 12pt
          \end{itemize}
     \item Outras tamanhos requerer o uso de pacotes adicionais!
  \end{itemize}

\end{frame}

%--------------------------------------------------------------------------%

\begin{frame}[fragile]

  \frametitle{Definindo novo tamanho de fonte}

  Para definir um tamanho de fonte precisa-se criar um comando!

  \begin{block}<1->{}
     \begin{minted}[fontsize=\scriptsize, resetmargins=true]{tex}
       \makeatletter
         \newcommand\tinyv{\@setfontsize\tinyv{6pt}{6pt}}
       \makeatother
     \end{minted}
  \end{block}

  \begin{block}<2->{Exemplo:}
    \begin{columns}
      \begin{column}[l]{5cm}
         \mint{tex}| \tinyv{Fonte ainda menor} |
      \end{column}

      \begin{column}[r]{0cm}
         $\rightarrow$
      \end{column}

      \begin{column}[r]{4cm}
        \tinyv{Fonte ainda menor}
      \end{column}
    \end{columns}
  \end{block}

\end{frame}

%--------------------------------------------------------------------------%

\begin{frame}[fragile]
  \frametitle{Estilos de Fonte}

  Diferentes estilos de fontes podem ser escolhidos para personalizar sua apresentação. \\

  Cada estilo de fonte está separada em um pacote diferente.
  \mint{tex}|\usepackage{helvet}|

  Para usar um estilo def fonte use o comando:
  \mint{tex}|\fontfamily{euler}|

  \begin{block}{Fontes disponíveis no Beamer}
    \begin{tabular}{l l l l l}
      serif & euler & newcent & avant & helvet \\
      palatino & bookman & mathtime & pifont & chancery \\
      mathptm & utopia & charter & mathptmx \\
    \end{tabular}
  \end{block}

\end{frame}
