\section{Elementos Matem\'aticos}
\subsection{Elementos b\'asicos}

\begin{frame}[fragile]{Elementos b\'asicos}
\begin{itemize}
\item Utiliza-se \verb|$...$| para produzir f\'ormulas dentro de um par\'agrafo;
\item Utiliza-se \verb|\[...]| para produzir equa{\c c}\~oes destacadas do par\'agrafo;
\item Utiliza-se \verb|\begin{equation}...\end{equation}| para poder referenciar a equa{\c c}\~ao usando \verb|\ref{}|.
\end{itemize}
\end{frame}


\begin{frame}[fragile]{Exemplos}

\verb|A Equação \ref{eqn:exemplo} é apresentada abaixo:|\newline
\verb|\newline| \newline
\verb|\begin{equation}\label{eqn:exemplo}|\newline
\verb|2x^2-3x+1=0|\newline
\verb|\end{equation}|\newline

Isso produz:\newline

A Equação \ref{eqn:exemplo} é apresentada abaixo: \newline
\begin{equation}\label{eqn:exemplo}
2x^2-3x+1=0
\end{equation}
\end{frame}


\begin{frame}[fragile]{Exemplos}

\verb|A Equação \ref{eqn:exemplo2} é apresentada abaixo:|\newline
\verb|\newline| \newline
\verb|\begin{equation}\label{eqn:exemplo2}|\newline
\verb|\frac{-b\pm\sqrt{b^2-4ac}}{2a}|\newline
\verb|\end{equation}|\newline

Isso produz:\newline

A Equação \ref{eqn:exemplo2} é apresentada abaixo: \newline
\begin{equation}\label{eqn:exemplo2}
x=\frac{-b\pm\sqrt{b^2-4ac}}{2a}
\end{equation}
\end{frame}



\begin{frame}[fragile]{Exemplos}
\begin{itemize}
\item \verb|$\sqrt[3]{8}=2$| produz $\sqrt[3]{8}=2$.
\item \verb|$a_n, x_i^2, x^{2n}$| produz $a_n, x_i^2, x^{2n}$.
\item \verb|$\int\limits_a^b f(x)dx$| produz $\int\limits_a^b f(x)dx$.
\item \verb|$\sum_{i=1}^n a_i$| produz $\sum_{i=1}^n a_i$.
\item \verb|$\sum\limits_{i=1}^n a_i$| produz $\sum\limits_{i=1}^n a_i$.
\item \verb|${n+1\choose k}={n\choose k}+{n\choose k-1}$| produz ${n+1\choose k}={n\choose k}+{n\choose k-1}$.
\end{itemize} 
\end{frame}



\begin{frame}[fragile]{Fun{\c c}\~oes Matem\'aticas}

Algumas das fun{\c c}\~oes pr\'e-definidas:
\newline \newline
\verb|\arccos \arcsin \arctan \arg \cos|
\verb|\cosh \cot \coth \csc \deg \det|
\verb|\dim \exp \gcd \hom \inf \ker \lg|
\verb|\lim \liminf \limsup \ln \log \max|
\verb|\min \Pr \sec \sin \sinh \sup \tan \tanh|

\end{frame}


\begin{frame}[fragile]{Matrizes}
Permite descrever tabelas e matrizes. Exemplo:
\begin{verbatim}
\begin{array}{clcr}
a+b+c & uv     & x-y & 27 \\
a+b   & u+v    & z   & 134 \\
a     & 3u+vw  & xyz & 2.978 \\ 
\end{array}
\end{verbatim}

Produz:

$\begin{array}{clcr}
$a+b+c$ & uv     & $x-y$ & 27 \\
$a+b$   & $u+v$    & z   & 134 \\
a     & $3u+vw$  & xyz & 2.978 \\

\end{array}$

\end{frame}


\begin{frame}[fragile]{Matrizes}
Matrizes podem ser obtidas usando-se delimitadores – \{, $[$, $($. Para indicar se o delimitador \'e o esquerdo ou o direito, deve-se anteceder o delimitador por \verb|\left| ou \verb|\right|. Exemplo:

\verb|\[ \left [| \newline
\verb|\begin{array}{clcr}| \newline
\verb|$a+b+c$ & uv     & $x-y$ & 27 \\| \newline
\verb|$a+b$   & $u+v$    & z   & 134 \\| \newline
\verb|a     & $3u+vw$  & xyz & 2.978 \\| \newline
\verb|\end{array}| \newline
\verb|\right ] \]| \newline

\[ \left [|
\begin{array}{clcr}
$a+b+c$ & uv     & $x-y$ & 27 \\
$a+b$   & $u+v$    & z   & 134 \\
a     & $3u+vw$  & xyz & 2.978 \\
\end{array}
\right ] \]

\end{frame}


\begin{frame}[fragile]{Exemplo}

\verb|\[ \left (| \newline
\verb|\begin{array}{ccc}| \newline
\verb|a_{11} & \cdots & a_{1n} \\| \newline
\verb|\vdots & \ddots & \vdots \\ | \newline
\verb|a_{m1} & \cdots & a_{mn}| \newline
\verb|\end{array}| \newline
\verb|\right ) \] | \newline

\[ \left (
\begin{array}{ccc}
a_{11} & \cdots & a_{1n} \\
\vdots & \ddots & \vdots \\
a_{m1} & \cdots & a_{mn}
\end{array} \right ) \]


\end{frame}
