\section{Hist\'orico}
\subsection{O que \'e \LaTeX?}


%----------------O que é Tex -------------------
\begin{frame}
\frametitle{O que \'e \TeX?}

\begin{itemize}

\item Criado originalmente por Donald E. Knuth (\TeX em 1977);
\item Desenvolvido para escrever livros com alta qualidade;
\item Knuth afirma que o \TeX n\~ao tem \textit{bugs};
\item O n\'umero da vers\~ao converge para $\pi$;
\item Pron\'uncia correta \'e "Tech" (no entanto existe a variante "Teks").

\end{itemize}

\end{frame}


%----------------O que é LaTex -------------------
\begin{frame}
\frametitle{O que \'e \LaTeX}

\begin{itemize}

\item Conjunto de macros que permitem a cria{\c c}\~ao de documentos com leioute pr\'e-definido;
\item Desenvolvido por Leslie Lamport;
\item O \LaTeX \'e um programa de c\'odigo aberto;
\item Existem v\'arias implementa{\c c}\~oes (TeTex, TexLive, MikTeX, etc);
\item A pron\'uncia correta \'e "Lay-Tech". Existem variantes como "Lah-Tech" e "Lah-Teks".

\end{itemize}

\end{frame}

%---------------- Vantagens -------------------
\begin{frame}
\frametitle{Vantagens}

\begin{itemize}

\item Resultado superior (melhor formata{\c c}~ao e qualidade tipogr\'afica);
\item Portabilidade, estabilidade e disponibilidade;
\item Seus documentos podem ser facilmente e corretamente estruturados;
\item \'Indices, notas de rodap\'e e referencias s\~ao geradas facilmente;
\item F\'ormulas matem\'aticas tamb\'em s\~ao facilmente criadas;

\end{itemize}
\end{frame}

%---------------- Desvantagens -------------------
\begin{frame}
\frametitle{Desvantagens}

\begin{itemize}

\item Uso de ferramentas auxiliares;
\item Trabalha-se diretamente com o "c\'odigo" e n\~ao o "visual" (com exce{\c c}\~oes);
\item \'E necess\'ario ter conhecimento dos comandos \LaTeX;
\item Em alguns momentos pode ser dif\'icil de conseguir alguns "\textit{looks}".

\end{itemize}
\end{frame}
