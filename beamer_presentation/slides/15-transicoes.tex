\section{Transições}
\subsection{Efeitos nos slides}
%------------------------------------------------------------

\begin{frame}[fragile]
  \frametitle{Transições de Slides}

  \begin{itemize}
     \item Com o Beamer também é possível definir efeitos transições de slides.
     \item Porém, diferentes visualizadores de pdf podem interpretar de diferentes maneiras os efeitos.
  \end{itemize}

  \begin{block}{Exemplo:}
    \begin{minted}[fontsize=\scriptsize]{tex}
        \begin{frame}
          \frametitle{Examplo Boxin }
          \transboxin
          Corpo do frame
        \end{frame}
    \end{minted}
  \end{block}

\end{frame}

%------------------------------------------------------------
\begin{frame}[fragile]
  \transboxin

  \begin{block}{Exemplo 2:}
    \begin{minted}[fontsize=\scriptsize]{tex}
        \begin{frame}
          \frametitle{Examplo blinds Horizontal }
          \transblindshorizontal[duration=2, direction=25]
          Corpo do frame
        \end{frame}
    \end{minted}
  \end{block}
\end{frame}

%-------------------------------------------------------

\begin{frame}[fragile]
  \transblindshorizontal[duration=2, direction=25]

  \begin{block}{Alguns opções de transições}
      \centering
      \begin{tabular}{l}
        $\backslash$transblindshorizontal \\
        $\backslash$transblindsvertical   \\
        $\backslash$transboxin \\
        $\backslash$transboxout \\
        $\backslash$transdissolve \\
        $\backslash$transglitter \\
        $\backslash$transslipverticalin \\
        $\backslash$transslipverticalout \\
        $\backslash$transhorizontalin \\
        $\backslash$transhorizontalout \\
        $\backslash$transwipe \\
        $\backslash$transduration\{2\} \\
      \end{tabular}
  \end{block}
\end{frame}
