\section{Tabelas}
\subsection{Construindo tabelas}

%-----------------------------------------------------------------------%

\begin{frame}[fragile]
  \frametitle{Tabelas}
   Tabelas são facilmente criadas com o comando \textit{tabular}

   \begin{itemize}
      \item Tabelas começam com o comando $\backslash$begin\{tabular\}\{ccc\}.
      \item \textbf{\{ccc\}} específica o número de colunas.
      \item Colunas são alinhas à direita \textbf{r}, à esquerda \textbf{l} e centralizado \textbf{c}
      \item Pode se ter vários alinhamentos na mesma tabela.
      \item O simbolo \& divide cada coluna.
      \item A \verb|\\| divide as linhas.
   \end{itemize}

\end{frame}

%-----------------------------------------------------------------------%

\begin{frame}[fragile]
  \frametitle{Exemplo 1:}

  \begin{columns}
    \begin{column}[c]{5cm}
      \begin{minted}[fontsize=\scriptsize]{tex}
       \begin{tabular}{ccc}
         cell 1 & cell 2 & cell 3 \\
         cell 4 & cell 5 & cell 6 \\
       \end{tabular}
     \end{minted}

    \end{column}
      \begin{tabular}{ccc}
        cell 1 & cell 2 & cell 3 \\
        cell 4 & cell 5 & cell 6 \\
      \end{tabular}
    \begin{column}[c]{2cm}

    \end{column}
  \end{columns}

  \vspace{10pt}
  \line(1,0){300}

  \begin{columns}[Código]
    \begin{column}[c]{5cm}
      \begin{minted}[fontsize=\scriptsize]{tex}
       \begin{tabular}{ccc}
         \hline
         cell 1 & cell 2 & cell 3 \\
         \hline
         cell 4 & cell 5 & cell 6 \\
         \hline
       \end{tabular}
     \end{minted}

    \end{column}
      \begin{tabular}{ccc}
        \hline
        cell 1 & cell 2 & cell 3 \\
        \hline
        cell 4 & cell 5 & cell 6 \\
        \hline
      \end{tabular}
    \begin{column}[c]{2cm}

    \end{column}
  \end{columns}
\end{frame}

%-----------------------------------------------------------------------%

\begin{frame}[fragile]
  \frametitle{Exemplo 2: Linhas Horizontais}
  Código:
  \begin{center}

     \begin{minted}[fontsize=\scriptsize]{tex}
       \begin{tabular}{|c|c|c|}
         cell 1 & cell 2 & cell 3 \\
         cell 4 & cell 5 & cell 6 \\
       \end{tabular}
     \end{minted}

  \end{center}

  Resultado:
  \begin{center}
     \begin{tabular}{|c|c|c|}
       cell 1 & cell 2 & cell 3 \\
       cell 4 & cell 5 & cell 6 \\
     \end{tabular}
  \end{center}

  A Tabela \ref{completa} possui linhas e colunas.

\end{frame}

%-----------------------------------------------------------------------%

\begin{frame}[fragile]
  \frametitle{Exemplo 3: Completa}

  \begin{minted}[fontsize=\tiny]{tex}
    \begin{table}
     \begin{tabular}{|c|c|c|}
       \hline
       \multicolumn{3}{|c|}{Tabela de Exemplo} \\
       \hline
       cell 1 & cell 2 & cell 3 \\
       \hline
       cell 4 & cell 5 & cell 6 \\
       \hline
     \end{tabular}
     \label{completa}
     \caption{Tabela com linhas e colunas}
    \end{table}
  \end{minted}
  \vspace{1cm}
  \begin{flushright}
    \begin{table}
     \begin{tabular}{|c|c|c|}
       \hline
       \multicolumn{3}{|c|}{Tabela de Exemplo} \\
       \hline
       cell 1 & cell 2 & cell 3 \\
       \hline
       cell 4 & cell 5 & cell 6 \\
       \hline
     \end{tabular}
     \label{completa}
     \caption{Tabela com linhas e colunas}
    \end{table}
  \end{flushright}

\end{frame}
