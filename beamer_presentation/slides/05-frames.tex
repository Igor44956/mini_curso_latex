\section{Frames}
\subsection{Definindo slides}

\begin{frame}
  \frametitle{Frames}

  \visible<1-2>{Um projeto Beamer é constituído de vários \textbf{frames}. Cada \textbf{frame} produz um ou mais
  slides dependendo do uso dos \textit{overlays}.}

  \vspace{1cm}

  \begin{block}<2->{Examplo de um frame básico}
    \inputminted[fontsize=\scriptsize]{tex}{codes/01-frame.tex}
  \end{block}

\end{frame}

%-------------------------------------------------------------------------------------------------------------%

\begin{frame}

  \begin{block}<1->{Alinhamento}
    A opção de alinhamento \textit{default} é centralizada \textbf{[c]}.
    Mas existe as opções \textbf{[t]} para o alinhamento no topo (\textit{top align}), e \textbf{[b]} para
    o alinhamento no rodapé (\textit{bottom align}).
  \end{block}

  \vspace{1cm}

  \begin{block}<2>{Alinhamento}
    \inputminted[fontsize=\scriptsize]{tex}{codes/02-frame.tex}
  \end{block}
\end{frame}

%-------------------------------------------------------------------------------------------------------------%

\begin{frame}[plain]
  \begin{itemize}[<+->]
     \item A opção \textbf{plain} \textcolor{red}{$\backslash$begin\{frame\}[plain]}
           retira do slide os cabeçalhos, rodapés, e diminui a barra de navegação.
     \item Este recurso é importante para apresentar figuras.
  \end{itemize}
\end{frame}

%-------------------------------------------------------------------------------------------------------------%

\begin{frame}[fragile]
  \frametitle{Frames especiais}

  \begin{itemize}
     \item \mint{tex}|\titlepage|
     \item \mint{tex}|\tableofcontents[\pausesections] |
  \end{itemize}

   \begin{minted}[fontsize=\scriptsize,resetmargins=true]{tex}
     ..
     \begin{frame}
       \titlepage
     \end{frame}
     ...
     \begin{frame}
       \frametitle{Sumario}
       \tableofcontents
     \end{frame}
     ...
  \end{minted}

\end{frame}

%---------------------------------------------------------------------------------------%

\begin{frame}[fragile]
  \frametitle{Título do Frame}

   \begin{minted}[fontsize=\scriptsize,resetmargins=true]{tex}
     \begin{frame}
       \frametitle{Aqui vai o titulo do frame}
     \end{frame}
   \end{minted}

\end{frame}

%---------------------------------------------------------------------------------------%

\begin{frame}
  \frametitle{Tudo junto - Slides Iniciais}
  \inputminted[linenos,fontsize=\scriptsize]{tex}{codes/04-template-inicial.tex}
\end{frame}

