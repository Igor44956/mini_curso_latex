\section{Refer\^encias Bibliogr\'aficas}
\subsection{Elementos b\'asicos}

\begin{frame}[fragile]{Elementos b\'asicos}
\begin{itemize}
\item BiB\TeX \'e um programa externo que permite definir refer\^encias bibliogr\'aficas;
\item Usa uma rela{\c c}\~ao de refer\^encias definidas em um arquivo \texttt{.bib};
\item S\~ao importadas somente as refer\^encias indicadas pelos comandos \verb|\cite| e \verb|\nocite|;
\item O programa \texttt{bibtex} l\^e o arquivo \texttt{.aux} gerado pelo \LaTeX;
\item O comando \verb|\bibliography{nome}| informa que a bibliografia encontra-se no arquivo \texttt{nome.bib};
\item O comando \verb|\bibliographystyle{...}| define o estilo da bibliografia a ser produzida (existem v\'arios estilos).
\end{itemize}
\end{frame}

\subsection{BiB\TeX}

\begin{frame}[fragile]{BiB\TeX}
Estrutura do arquivo .bib: cont\'em uma sequ\^encia de entradas, sendo cada entrada definida como:\newline
\newline 
\verb|@tipo{rótulo, chave={valor}, chave={valor},...}|
\newline \newline
Tipos de entradas mais comuns:
\newline
\begin{description}
\item[book] livro;
\item[inproceedings] artigo em anais de evento;
\item[article] artigo em peri\'odico.
\end{description}
\end{frame}

\begin{frame}[fragile,allowframebreaks]{BiB\TeX: Exemplo}

\verb|@Book{livropca,| \newline
\verb|author = {Ian T. Jolliffe},| \newline
\verb|publisher = {Springer-Verlag},| \newline
\verb|title = {Principal Component Analysis},| \newline
\verb|year = {2002},| \newline
\verb|note = {ISBN 0387954422}| \newline
\verb|}| \newline

\cite{latex_wikibooks}

\end{frame} 
