\section{Funcionamento}
\subsection{Exemplos b\'asicos}

\begin{frame}[fragile]{Exemplo b\'asico}
\verb| % Este é um pequeno arquivo fonte para o LaTeX| \\
\verb| % O símbolo "%" indica um comentário e é ignorado| \\ 
\verb| \documentclass[10pt]{article}| \\
\verb| \usepackage[latin1]{inputenc}| \\
\verb| \usepackage[brazil]{babel}| \\
\verb| \usepackage{graphicx}| \\
\verb| \begin{document}| \\
\verb| Meu primeiro texto| \\
\verb| \section{Minha primeira seção}| \\
\verb| \end{document}|
\end{frame}



\begin{frame}[fragile]{Sem reinventar a roda: pacotes}
O \LaTeX fornece muitos recursos pr\'e-instalados, por meio de pacotes. Esses pacotes fornecem uma variedade enorme de recursos configurados, havendo a necessidade apenas de conhecer suas caracter\'isticas (forma correta de utiliza{\c c}\~ao).
\newline
\newline
Mas como verificar a documenta{\c c}\~ao de um pacote? No linux, apenas digitar no terminal "texdoc nome-do-pacote".
\newline
\newline
Nas nossas refer\^encias est\~ao links para alguns dos muitos pacotes, dentre os mais utilizados.

\end{frame}


\begin{frame}[fragile]{Mudando o tamanho da fonte}
Pode-se alterar facilmente o tamanho da fonte utilizada nos textos. O formato padr\~ao do comando \'e \verb|\fonte{texto}|. Exemplos de formatos de fontes:\\
\begin{description}
\item[tiny]: \tiny{texto em tiny};
\item[scriptsize]: \scriptsize{texto em scriptsize};
\item[footnotesize]: \footnotesize{texto em footnotesize}; 
\item[small]: \small{texto em small};
\item[normalsize]: \normalsize{texto em normalsize};
\item[Large]: \Large{texto em Large};
\item[huge]: \huge{texto em huge}.
\end{description}
\end{frame}


\begin{frame}[fragile]{Definindo divis\~oes no texto}
Divis\~oes pr\'e-definidas em \LaTeX:\\
\verb|\part| \\
\verb|\chapter| \\
\verb|\section| \\
\verb|\subsection|\\
\verb|\subsubsection|\\
\verb|\paragraph| \\
\verb|\subparagraph|\\

\begin{itemize}
\item O estilo \textit{article} n\~ao permite o comando \verb|\chapter|;
\item A numera{\c c}\~ao de cap\'itulos/se{\c c}\~oes/subse{\c c}\~oes \'e gerada automaticamente pelo \LaTeX.
\end{itemize}
\end{frame}
