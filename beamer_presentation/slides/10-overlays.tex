\section{Overlays}
\subsection{Criando efeitos no texto}

%-----------------------------------------------------------------------%

\begin{frame}
  \frametitle{Overlays}

  \begin{itemize}[<+->]
     \item São usado para partes da apresentação aparecerem incrementalmente.
     \item No Beamer, os \textcolor{red}{overlays} controlam que partes do frame irá aparecer.
  \end{itemize}

\end{frame}

%-----------------------------------------------------------------------%

\begin{frame}[fragile]
  \frametitle{Overlays}

  \begin{block}<1->{}
    Uma das maneiras de fazer partes dos texto aparecerem separados é utilizando o comando
    \mint{tex}| \pause |
  \end{block}

  \begin{block}<2->{Exemplo do pause:}
    \inputminted[fontsize=\scriptsize]{tex}{codes/08-pause-command.tex}
  \end{block}

  \begin{block}{}
     \pause
     \textbf{Passo 1:} Aprender Latex. \\
     \pause
     \textbf{Passo 2:} Aprender Beamer. \\
     \pause
     \textbf{Passo 3:} Não se preocupar com formatação!! =) \\
  \end{block}

\end{frame}

%-----------------------------------------------------------------------%

\begin{frame}[fragile]
  \frametitle{Especificações para os overlays}

  \begin{itemize}[<+->]
     \item Em overlays avançadas, muitos comandos e bloco de instruções incorporam algumas
           especificações.
     \item Para entender overlays primeiro temos que entender como a apresentação é mostrada.
     \item Para dar a ilusão do overlays usamos múltiplos slides com diferentes partes de texto.
     \item Por exemplo: \textbf{$\backslash$pause} criou 3 slides para realizar o efeito no texto.
  \end{itemize}

\end{frame}

%-----------------------------------------------------------------------%

\begin{frame}[fragile]
  \frametitle{Especificações para Overlays}

  \begin{itemize}
     \item As especificações de overlay são especificadas entre \verb|(<,>)| e indica que parte do slide
           deve aparecer.
     \item A especificação \verb|<1->|, significa mostre do slide 1 em diante.
     \item \verb|<1-3>|mostre do slide 1 até o 3.
     \item \verb|<-3, 5-6, 8->|, mostre todos os slides exceto o 4 e o 5.
  \end{itemize}
\end{frame}

%-----------------------------------------------------------------------%

\begin{frame}[fragile]
  \frametitle{Especificações Overlays}

    \begin{columns}
      \begin{column}[c]{5cm}
        \begin{minted}[fontsize=\scriptsize]{tex}
          \begin{itemize}
            \item<1>   Texto ... <1>
            \item<1-2> Texto ... <1-2>
            \item<1-2> Texto ... <1-2>
            \item<1>   Texto ... <1>
            \item<1,3> Texto ... <1,3>
            \item<1-2> Texto ... <1-2>
           \end{itemize}
       \end{minted}

      \end{column}

      \begin{column}[c]{5cm}
         \begin{itemize}
          \item<1>   Texto ... <1>
          \item<1-2> Texto ... <1-2>
          \item<1-2> Texto ... <1-2>
          \item<1>   Texto ... <1>
          \item<1,3> Texto ... <1,3>
          \item<1-2> Texto ... <1-2>
        \end{itemize}
      \end{column}
    \end{columns}

  \begin{block}<4->{Nota:}{\small
    Se você quiser que os itens apareçam na ordem que de listagem use a opção
    [<+->]. ( i.e. $\backslash$begin\{itemize\}[<+->]| )}
  \end{block}

\end{frame}

%-----------------------------------------------------------------------%

\begin{frame}[fragile]
  \frametitle{Especificações Overlays}
  As especificações Overlays também podem ser usadas para que certos comandos
  tenham efeito em diferentes times.

  \begin{minted}[fontsize=\scriptsize]{tex}
    \alert{Alerta para todos os slides}     \\
    \textcolor<2>{blue}{Azul no slide 2}    \\
    \textit<3>{Italico no slide 3}          \\
    \alert<1,3>{Alerta nos slides 1 e 3}    \\
  \end{minted}

  \begin{block}{Comandos}
    \inputminted[fontsize=\scriptsize]{tex}{codes/09-overlays.tex}
  \end{block}

\end{frame}

%-----------------------------------------------------------------------%

\begin{frame}[fragile]
  \frametitle{Especificações Overlays}

    \alert{Alerta para todos os slides}     \\
    \textcolor<2>{blue}{Azul no slide 2}    \\
    \textit<3>{Itálico no slide 3}          \\
    \alert<1,3>{Alerta nos slides 1 e 3}    \\

\end{frame}

%-----------------------------------------------------------------------%

\begin{frame}[fragile,plain]
  \frametitle{Especificações Overlays}
   Comandos especiais:

  \begin{tabular}{|p{4cm}|p{6cm}|}
    \hline
    \mint{tex}|\onslide<1,2>| & $\backslash$only<1-2>\{O texto dado como argumento apenas aparece nos slides especificados.
                                Se nenhum texto for passado qualquer texto após o comando aparecerá apenas
                                nos slides específicados\} \\
    \hline
    \mint{tex}|\visible| & $\backslash$visible<3>\{Este texto será visível somente no slide 3.\} \\
    \hline
    \mint{tex}|\invisible| & $\backslash$invisible<-2>\{Este texto fica invisível até o slide 2 e visível no restante.\}\\
    \hline
    \mint{tex}|\alt| & $\backslash$alt<2>\{Texto para o slide 2.\}\{Texto para o restante.\}\\
    \hline
  \end{tabular}

\end{frame}

%-----------------------------------------------------------------------%

\begin{frame}[fragile]
  \frametitle{Transparete Invisível}

  Os comandos abaixo habilitam a transparência e retorna ao comportamento normal de
  e esconder os itens nos frames.

  \begin{minted}[fontsize=\scriptsize]{language}
    ...
    \setbeamercovered{transparent}
    ...
    \setbeamercovered{invisible}
    ...
  \end{minted}

\end{frame}

%-----------------------------------------------------------------------%
\begin{frame}{Overlays com blocos}

  \setbeamercovered{invisible}

  \begin{block}<1->{Primeiro bloco}
    Este é o primeiro bloco
  \end{block}
  \vspace{1cm}
  \begin{block}<2->{Segundo bloco}
    Este é o segundo bloco \only<1>{transparente}
  \end{block}

  \setbeamercovered{transparent}
\end{frame}

%-----------------------------------------------------------------------%
